\documentclass[a4paper,10pt]{article}
\usepackage[utf8]{inputenc}
\usepackage{ragged2e}
\usepackage{graphicx}
\usepackage{hyperref}
\usepackage{enumerate}
\usepackage{dirtree}
\usepackage{fancyhdr}
\usepackage{listings}
\usepackage{scrextend}
\pagestyle{fancy}
\newcommand\tab[1][1cm]{\hspace*{#1}}
\title{Inferencia con Lógica de Predicados\\(6ª semana)}
\begin{document}
\maketitle
\pagebreak
\tableofcontents
\pagebreak


\section{ejercicio 9.4}
\subsection{Enunciado}
For each pairofatomicsentences, give the most general unifierif  it exist:
\begin{enumerate}
	\item P(A,B,B),P(x,y,z).
	\item Q(Y,G(A,B)),Q(G(x,x)y).
	\item Older(Father(y),Y),Older(Father(x),John).
	\item Knows(Father(y),Y),Knows(x,X).
\end{enumerate}
\subsection{Resolución}
\begin{itemize}
	\item	P(A,B,B),P(x,y,z).
	\begin{itemize}
		\item P(A,B,B) $\land$ P(x,y,z).
		\item P(A,B,B)\\P(x,y,z)
	\end{itemize}
	\item Q(Y,G(A,B)),Q(G(x,x)y).
	\begin{itemize}
		\item Q(Y,G(A,B)) $\land$ Q(G(x,x)y).
		\item Q(Y,G(A,B))\\Q(G(x,x)y).
	\end{itemize}
\end{itemize}


\vspace{1cm}
\section{ejercicio 6}
\subsection{Enunciado}
Expresa   en   sentencias   de   lógica   de   predicados   las   siguientes   ideas   y   comprueba   que los razonamientos son correctos utilizando la prueba por refutación y el principio de resolución.
\subsection{Resolución}
\begin{enumerate}
	\item Todas las personas no son altas. Todos los españoles son personas. Por tanto, todos los españoles no son altos.
	\begin{itemize}
		\item $\lnot$ $\forall$ persona $\rightarrow$ alta(persona) $\land$ $\forall$ español esPersona(español)
		\item simplificando a la forma normal:
		\item $\exists$ $\lnot$persona $\rightarrow$ alta(persona) $\land$ $\forall$ español esPersona(español)
		\item ($\exists$ persona $\lor$ alta(persona)) $\land$ $\forall$ español esPersona(español)
		\item (Persona $\lor$ alta(Persona)) $\land$ $\forall$ español esPersona(español)
		\item (Persona $\lor$ alta(Persona)) $\land$ español esPersona(español)
		\item (Persona $\lor$ alta(Persona)), español esPersona(español)
		\item por lo tento el razoniento es correcto
	\end{itemize}
	\item Todos los mamíferos tienen pulmones. Losárboles no tienen pulmones. Por tanto, losárbolesno son mamíferos.
	\begin{itemize}
		\item $\forall$ mamiferos, arboles (pulmones(mamiferos) $\land$ $\lnot$pulmones(arboles)) $\rightarrow$ $\lnot$mamiferos(arboles)
		\item simplificando a la forma normal:
		\item $\forall$ mamiferos, arboles $\lnot$(pulmones(mamiferos) $\land$ $\lnot$pulmones(arboles)) $\lor$ $\lnot$mamiferos(arboles)
		\item $\forall$ mamiferos, arboles ($\lnot$pulmones(mamiferos) $\lor$ pulmones(arboles)) $\lor$ $\lnot$mamiferos(arboles)
		\item ($\lnot$pulmones(mamiferos) $\lor$ pulmones(arboles)) $\lor$ $\lnot$mamiferos(arboles)
		\item Reducimos a lo absurdo
		\item $\lnot$(($\lnot$pulmones(mamiferos) $\lor$ pulmones(arboles)) $\lor$ $\lnot$mamiferos(arboles))
		\item ($\lnot$($\lnot$pulmones(mamiferos) $\lor$ pulmones(arboles)) $\land$ mamiferos(arboles))
		\item (pulmones(mamiferos) $\land$ $\lnot$pulmones(arboles)) $\land$ mamiferos(arboles))
		\item pulmones(mamiferos) $\land$ $\lnot$pulmones(arboles) $\land$ mamiferos(arboles))
		\item Nos da contradicción asi que el enunciado es valido
	\end{itemize}
	\item Los planetas giran alrededor del Sol. La Tierra es un planeta. Por tanto, la Tierra giraalrededor del Sol.
	\begin{itemize}
		\item $\forall$ planeta (girar(planeta, Sol) $\land$ planeta(Tierra)) $\rightarrow$ girar(Tierra, Sol)
		\item simplificando a la forma normal:
		\item $\lnot$$\forall$ planeta (girar(planeta, Sol) $\land$ planeta(Tierra)) $\lor$ girar(Tierra, Sol)
		\item $\exists$ planeta ($\lnot$girar(planeta, Sol) $\lor$ $\lnot$planeta(Tierra)) $\lor$ girar(Tierra, Sol)
		\item ($\lnot$girar(Planeta, Sol) $\lor$ $\lnot$planeta(Tierra)) $\lor$ girar(Tierra, Sol)
		\item $\lnot$girar(Planeta, Sol) $\lor$ $\lnot$planeta(Tierra) $\lor$ girar(Tierra, Sol)
		\item reducimos a lo absurdo
		\item $\lnot$($\lnot$girar(Planeta, Sol) $\lor$ $\lnot$planeta(Tierra) $\lor$ girar(Tierra, Sol))
		\item ($\lnot$($\lnot$girar(Planeta, Sol) $\lor$ planeta(Tierra)) $\land$ $\lnot$girar(Tierra, Sol)
		\item ((girar(Planeta, Sol) $\land$ $\lnot$planeta(Tierra)) $\land$ $\lnot$girar(Tierra, Sol)
		\item girar(Planeta, Sol) $\land$ $\lnot$planeta(Tierra) $\land$ $\lnot$girar(Tierra, Sol)
	\end{itemize}
	\item Todos   los   marineros   aman   el   mar.   Algunos   cordobeses   son   marineros.   Por   tanto,   algunoscordobeses aman el mar.
	\begin{itemize}
		\item $\forall$ mamiferos aman(mamiferos, Mar) $\land$ $\exists$ cordobés marinero(cordobés) $\rightarrow$ aman(cordobes, Mar)
		\item simplificando a la forma normal:
		\item $\forall$ mamiferos $\exists$ cordobés (aman(mamiferos, Mar) $\land$ marinero(cordobés)) $\rightarrow$ aman(cordobes, Mar)
		\item $\forall$ mamiferos (aman(mamiferos, Mar) $\land$ marinero(Cordobés)) $\rightarrow$ aman(Cordobes, Mar)
		\item aman(mamiferos Mar) $\land$ marinero(Cordobés) $\rightarrow$ aman(Cordobes, Mar)
		\item $\lnot$aman(mamiferos, Mar) $\lor$ $\lnot$marinero(Cordobés) $\lor$ aman(Cordobes, Mar)
		\item aplicamos reduccion a lo absurdo
		\item $\lnot$(($\lnot$aman(mamiferos, Mar) $\lor$ $\lnot$marinero(Cordobés)) $\lor$ aman(Cordobes, Mar))
		\item $\lnot$($\lnot$aman(mamiferos, Mar) $\lor$ $\lnot$marinero(Cordobés)) $\land$ $\lnot$aman(Cordobes, Mar)
		\item (aman(mamiferos, Mar) $\land$ marinero(Cordobés)) $\land$ $\lnot$aman(Cordobes, Mar)
		\item aman(mamiferos, Mar) $\land$ marinero(Cordobés) $\land$ $\lnot$aman(Cordobes, Mar)
		\item La conclusion no es conclusiva
	\end{itemize}
	\item Los ingleses hablan inglés. Los españoles no son ingleses. Algunos españoles hablan inglés. Por tanto, algunos que hablan inglés no son ingleses.
	\begin{itemize}
		\item $\forall$ ingles $\exists$ español $\lnot$habla(ingles, Ingles) $\lor$ $\lnot$habla(español, Español) $\rightarrow$ habla(español, ingles)
		\item simplificando a la forma normal:
		\item $\forall$ ingles $\exists$ español $\lnot$habla(ingles, Ingles) $\lor$ $\lnot$habla(español, Español) $\lor$ habla(español, ingles)
		\item $\forall$ ingles $\lnot$habla(ingles, Ingles) $\lor$ $\lnot$habla(Español, Español) $\lor$ habla(Español, ingles)
		\item $\lnot$habla(ingles, Ingles) $\lor$ $\lnot$habla(Español, Español) $\lor$ habla(Español, ingles)
		\item Aplicamos reducción a lo absurdo.
		\item habla(ingles, Ingles) $\land$ habla(Español, Español) $\land$ $\lnot$habla(Español, ingles)
	\end{itemize}
	\item Ningún mamífero tiene sangre fría. Los peces tienen sangre fría. Los peces viven en el agua ynadan. Algunos mamíferos viven en el agua y nadan. Las ballenas tienen sangre caliente. Portanto, las ballenas son mamíferos.
	\begin{itemize}
		\item $\lnot$$\exists$ mamifero1 $\exists$ mamifero2 $\forall$ pez $\forall$ ballena ( sangreFria(mamifero1) $\land$ sangreFria(pez) $\land$ vivir(pez, Agua) ) $\land$ ( vivir(mamifero2, Agua) $\land$ nada(mamifero2) ) $\land$ ( $\lnot$sangreFria(ballena) ) $\rightarrow$ mamirefo(ballena)
		\item simplificando a la forma normal:
		\item $\lnot$$\exists$ mamifero1 $\exists$ mamifero2 $\forall$ pez $\forall$ ballena\\
		$\lnot$(( sangreFria(mamifero1) $\land$ sangreFria(pez) $\land$ vivir(pez, Agua) ) $\land$ ( vivir(mamifero2, Agua) $\land$ nada(mamifero2) ) $\land$ ( $\lnot$sangreFria(ballena) )) $\lor$ mamirefo(ballena)
		\item $\lnot$$\exists$ mamifero1 $\exists$ mamifero2 $\forall$ pez $\forall$ ballena\\
		( $\lnot$sangreFria(mamifero1) $\lor$ $\lnot$sangreFria(pez) $\lor$ $\lnot$vivir(pez, Agua) )\\
		$\lor$ ($\lnot$ vivir(mamifero2, Agua) $\lor$ $\lnot$nada(mamifero2) )\\
		$\lor$ ( sangreFria(ballena) ))\\
		$\lor$ mamirefo(ballena)\\
		\item $\forall$ mamifero1 $\exists$ mamifero2 $\forall$ pez $\forall$ ballena\\
		( sangreFria(mamifero1) $\lor$ $\lnot$sangreFria(pez) $\lor$ $\lnot$vivir(pez, Agua) )\\
		$\lor$ ($\lnot$ vivir(mamifero2, Agua) $\lor$ $\lnot$nada(mamifero2) )\\
		$\lor$ ( sangreFria(ballena) ))\\
		$\lor$ mamirefo(ballena)\\
		\item ( sangreFria(mamifero1) $\lor$ $\lnot$sangreFria(pez) $\lor$ $\lnot$vivir(pez, Agua) )\\
		$\lor$ ($\lnot$ vivir(mamifero2, Agua) $\lor$ $\lnot$nada(mamifero2) )\\
		$\lor$ ( sangreFria(ballena) ))\\
		$\lor$ mamirefo(ballena)\\
		\item ( sangreFria(mamifero1) $\lor$ $\lnot$sangreFria(pez) $\lor$ $\lnot$vivir(pez, Agua) )\\
		$\lor$ ($\lnot$ vivir(mamifero2, Agua $\lor$ $\lnot$nada(mamifero2) )\\
		$\lor$ ( sangreFria(ballena) ))\\
		$\lor$ mamirefo(ballena)\\
		\item Aplicamos reducción a lo absurdo.
		\item $\lnot$( ( sangreFria(mamifero1) $\lor$ $\lnot$sangreFria(pez) $\lor$ $\lnot$vivir(pez, Agua) )\\
		$\lor$ ($\lnot$ vivir(mamifero2, Agua $\lor$ $\lnot$nada(mamifero2) )\\
		$\lor$ ( sangreFria(ballena) ))\\
		$\lor$ mamirefo(ballena) )\\
		\item $\lnot$( sangreFria(mamifero1) $\lor$ $\lnot$sangreFria(pez) $\lor$ $\lnot$vivir(pez, Agua) )\\
		$\land$ $\lnot$($\lnot$ vivir(mamifero2, Agua $\lor$ $\lnot$nada(mamifero2) )\\
		$\land$ $\lnot$( sangreFria(ballena) ))\\
		$\land$ $\lnot$mamirefo(ballena)\\
		\item ( $\lnot$sangreFria(mamifero1) $\land$ sangreFria(pez) $\land$ vivir(pez, Agua) )\\
		$\land$ ( vivir(mamifero2, Agua $\land$ nada(mamifero2) )\\
		$\land$ ($\lnot$sangreFria(ballena) )) $\land$ $\lnot$mamirefo(ballena)\\
		\item $\lnot$sangreFria(mamifero1) $\land$ sangreFria(pez) $\land$ vivir(pez, Agua) \\
		$\land$ vivir(mamifero2, Agua $\land$ nada(mamifero2) \\
		$\land$ $\lnot$sangreFria(ballena) $\land$ $\lnot$mamirefo(ballena)\\
		\item $\lnot$sangreFria(mamifero1) $\land$ $\lnot$sangreFria(ballena) $\land$ sangreFria(pez) \\
		$\land$ vivir(pez, Agua) $\land$ vivir(mamifero2, Agua)\\
		$\land$ nada(mamifero2) $\land$ $\lnot$mamirefo(ballena)\\
	\end{itemize}
	\item Si el reloj estaba adelantado, Juan llegóantes de las diez y vio partir el coche de Andrés. Si Andrés   dice   la   verdad   entonces,   Juan   no   vio   partir   el   coche   de   Andrés.   O   Andrés   dice   laverdad o estaba en el edificio en el momento del crimen. El reloj estaba adelantado. Por tanto,Andrés estaba en el edificio en el momento del crimen.
	\item Pepito  recibe   regalos  en  su  cumpleaños  y  en  su  santo.  Pepito   no  recibióregalos   ayer.  Portanto, ayer no fue su cumpleaños ni su santo.
	\item Marta va al cine siempre que tiene dinero o alguien le invita, y sólo en esos
	casos. Marta fuéayer al cine y nadie le invitó. Por tanto, Marta tenía dinero ayer.
	\item Uno es adorable si y sólo si todo el mundo lo ama. Pepito no es adorable. Por tanto, alguien noama a Pepito.

\end{enumerate}


\vspace{1cm}
\section{ejercicio 7}
\subsection{Enunciado}
Pasa a forma normal clausulada las siguientes fórmulas:
\subsection{Resolución}
\begin{enumerate}
	\item $\exists$ x $\exists$ y p(x, y)
	\begin{itemize}
		\item p(X, Y)
	\end{itemize}
	\item $\forall$ x $\exists$ y p(x, y)
	\begin{itemize}
		\item $\forall$ x p(x, Y)
		\item p(x, Y)
	\end{itemize}
	\item $\exists$ x $\forall$ y p(x, y)
	\begin{itemize}
		\item $\forall$ y p(X, y)
		\item p(X, y)
	\end{itemize}
	\item $\forall$ x $\forall$ y p(x, y)
	\begin{itemize}
		\item p(x, y)
	\end{itemize}
	\item $\forall$ x $\exists$ y (p(x,y) $\rightarrow$ p(y,x))
	\begin{itemize}
		\item $\forall$ x $\exists$ y $\lnot$(p(x,y) $\lor$ p(y,x))
		\item $\forall$ x $\lnot$p(x,Y) $\lor$ p(Y,x)
		\item $\lnot$p(x,Y) $\lor$ p(Y,x)
	\end{itemize}
	\item $\forall$ x ($\exists$ y p(y,x) $\rightarrow$ $\forall$ x $\exists$ z $\lnot$ q(x, z))
	\begin{itemize}
		\item $\forall$ x (p(Y,x) $\rightarrow$ $\forall$ x $\lnot$ q(x, Z))
		\item $\forall$ x0 (p(Y,x0) $\rightarrow$ $\forall$ x1 $\lnot$ q(x1, Z))
		\item $\forall$ x0 $\lnot$(p(Y,x0) $\lor$ $\forall$ x1 $\lnot$ q(x1, Z))
		\item $\lnot$(p(Y,x0) $\lor$ $\lnot$ q(x1, Z))
		\item $\lnot$p(Y,x0) $\land$ q(x1, Z)
		\item $\lnot$p(Y,x0), q(x1, Z)
	\end{itemize}
	\item ($\forall$ x p(x)) $\rightarrow$ [ $\forall$ x $\forall$ y $\exists$ z (q(x, y, z) $\rightarrow$ r(x, y, z, u))]
	\begin{itemize}
		\item ($\forall$ x0 p(x0)) $\rightarrow$ [ $\forall$ x1 $\forall$ y $\exists$ z (q(x1, y, z) $\rightarrow$ r(x1, y, z, u))]
		\item ($\forall$ x0 $\lnot$p(x0)) $\lor$ $\lnot$[ $\forall$ x1 $\forall$ y $\exists$ z $\lnot$(q(x1, y, Z) $\lor$ r(x1, y, Z, u))]
		\item ($\forall$ x0 $\lnot$p(x0)) $\lor$ $\lnot$[ $\forall$ x1 $\forall$ y $\lnot$(q(x1, y, Z) $\lor$ r(x1, y, Z, u))]
		\item ($\lnot$p(x0)) $\lor$ $\lnot$[ $\lnot$(q(x1, y, Z) $\lor$ r(x1, y, Z, u))]
		\item ($\lnot$p(x0)) $\lor$ [ (q(x1, y, Z) $\land$ $\lnot$r(x1, y, Z, u))]
		\item $\lnot$p(x0) $\lor$ [ q(x1, y, Z) $\land$ $\lnot$r(x1, y, Z, u)]
		\item ($\lnot$p(x0) $\lor$ q(x1, y, Z)) $\land$ ($\lnot$p(x0) $\lor$ $\lnot$r(x1, y, Z, u))
		\item $\lnot$p(x0) $\lor$ q(x1, y, Z), $\lnot$p(x0) $\lor$ $\lnot$r(x1, y, Z, u)
	\end{itemize}

\end{enumerate}


\vspace{1cm}
\section{ejercicio 8}
\subsection{Enunciado}
Dados los siguientes literales, indica si se pueden unificar o no:
\subsection{Resolución}
\begin{enumerate}
	\item p(x1,a) y p(b,x2)
	\\x1=b, x2=a
	\item p(x1,y1,f(x1,y1)) y p(x2,y2,g(a,b))
	\\no se puede (dos valores distintos para la misma variable)
	\item p(x1,a,f(a,b)) y p(c,y2,f(x2,b))
	\\x1=c, y2=a, x2=a
	\item p(f(a),g(x1)) y p(y2,y2)
	\\no se puede (dos valores distintos para la misma variable)
	\item p(f(a),g(x1)) y p(y2,z2)
	\\no se puede (se tendria que asignar que z2=g(x1))
\end{enumerate}

% \vspace*{\fill} %se va al final de  la pagina
% \raggedleft Documento escrito en \LaTeX{}
\end{document}
