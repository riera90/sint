\documentclass[a4paper,10pt]{article}
\usepackage[utf8]{inputenc} %Codificacion utf-8
\usepackage{graphicx}
\usepackage{enumerate}
\usepackage{fancyhdr}
\usepackage{hyperref}
\usepackage{tikz}     % graphs!
\usepackage{listings} % code!
\usepackage{multirow} % Required for multirows
\usepackage[spanish, activeacute]{babel} %Definir idioma español
% \usepackage[margin=3cm]{geometry}
\setlength{\headheight}{15pt}
\hypersetup{
    colorlinks=true,
    linkcolor=blue,
    filecolor=magenta,
    urlcolor=cyan,
}
\pagestyle{fancy}

\lhead{Técnicas de Búsqueda con Adversarios}
\rfoot{Página \thepage}
\lfoot{Sistemas Inteligentes}
\cfoot{}

\newcommand\tab[1][1cm]{\hspace*{#1}}

\title{Técnicas de Búsqueda con Adversarios\\(4ª semana)}
\author{}

\begin{document}

\maketitle
\pagebreak
\tableofcontents
\pagebreak

\section{La cena}
\subsection{Enunciado}
Resuelve el siguiente problema tanto por minimax como poda alfa-beta, añadiendo junto a cada nodo la información como se indica en el apartado anterior.
\paragraph{}
Tu hermano mayor y tú, vais a cenar. Vuestra madre ha preparado dos platos: coles de bruselas y brócoli, y os ha dicho que os lo comáis todo. Decidís que os vais a repartir los platos de la siguiente forma: primero, tú eliges entre comer uno de los dos platos o echarte la mitad de uno de ellos; después, tu hermano puede realizar las mismas elecciones sobre lo que quede en la mesa; así, mientras quede algo en la mesa, os iréis alternando en la elecciones (según este esquema, puede que alguno acabe comiendo más comida que el otro). Tú has cometido el error de hacerle saber tus preferencias a tu hermano, que son las que se muestran en la siguiente tabla. Sabiendo que el objetivo de tu hermano es únicamente fastidiarte, ¿qué plato debes elegir primero?
\paragraph{}
Nota 1: Se deben seguir las siguientes prioridades entre las reglas para quien tenga que elegir: 
\begin{itemize}
    \item [1)]escoger el plato de coles.
    \item [2)]escoger el plato de brócoli.
    \item [3)]escoger medio plato de coles.
    \item [4)]escoger medio plato de brócoli.
\end{itemize}
\paragraph{}
Nota 2: Poda alfa/beta puede ser más rápido y obtiene la máxima puntuación.

plato entero de coles 	-10\\
plato entero de brócoli 	0\\
plato entero de coles + medio plato de brócoli 	-15\\
plato entero de brócoli + medio plato de coles 	-5\\
medio plato de brócoli + medio plato de coles 	+5 (por probar un poco de todo)

\subsection{Resolución}



\end{document}
